%-------------------------------------------------------------------------------
%	SECTION TITLE
%-------------------------------------------------------------------------------
\cvsection{Projects}


%-------------------------------------------------------------------------------
%	CONTENT
%-------------------------------------------------------------------------------
\begin{cventries}

%---------------------------------------------------------
  \cventry
    {\href{https://lightspeed-triggers.vercel.app}{lightspeed-triggers.vercel.app}} % Job title
    {Ecommerce Store} % Organization
    {January 2023 - Present} % Location
    {} % Date(s)
    {
      \begin{cvitems} % Description(s) of tasks/responsibilities
        \item {Used Next.js to create a full-stack application connected to a MySQL database with the Prisma ORM as an abstraction over the database.}
        \item {Created a performant SEO-optimized landing page.}
        \item {Implemented CRUD functionality on the back end to allow the site administrator to create customization options for their users.}
        \item {Integrated with Stripe to securely handle purchases.}
        \item {Deployed the project to a Linux server through SSH.}
        \item {Created an admin-only authentication system with NextAuth and ensured that the backend was inaccessible to ordinary users.}
        \item {Used OpenSSL to create cryptographically secure passwords and environment variables.}
      \end{cvitems}
    }

%---------------------------------------------------------

  \cventry
  {\href{https://github.com/ryangroch/ComEdPrices}{github.com/ryangroch/ComEdPrices}} % Job title
  {ComEd Prices} % Organization
  {December 2023} % Location
  {} % Date(s)
  {
    \begin{cvitems} % Description(s) of tasks/responsibilities
      \item {Used the Tauri framework to write a desktop application with TypeScript.}
      \item {Used with the official ComEd API to retrieve and display electricity prices.}
      \item {Configured GitHub Actions to automatically generate binaries for Windows, MacOS, and Debian-based Linux distributions.}
    \end{cvitems}
  }

% ---------------------------------------------------------

% \cventry
% {\href{https://ryan-groch.github.io/image-cropper/}{ryan-groch.github.io/image-cropper/}} % Job title
% {Image Cropper} % Organization
% {March 2023} % Location
% {} % Date(s)
% {
%   \begin{cvitems} % Description(s) of tasks/responsibilities
%     \item {Wrote an algorithm in TypeScript to crop an image in a circular pattern.}
%     \item {Utilized HTML canvas for image processing.}
%     \item {Used GitHub Actions for automatic deployment.}
%   \end{cvitems}
% }

%---------------------------------------------------------
  \cventry
    {\href{https://ryangroch.github.io/othello-ai-v2/}{ryangroch.github.io/othello-ai-v2/}} % Job title
    {Othello AI} % Organization
    {April 2022 - November 2022} % Location
    {} % Date(s)
    {
      \begin{cvitems} % Description(s) of tasks/responsibilities
        \item {Implemented the rules of the Othello game in Typescript, and created a UI with React.js which tracks the state of the game.}
        \item {Used the minimax algorithm with alpha-beta pruning to create an AI that demonstrates a reasonable quality of play.}
        \item {Made use of web workers and asynchronous programming to avoid freezing the UI as the algorithm performs computations.}
        \item {Used GitHub actions to automatically redeploy the app on each new commit.}
      \end{cvitems}
    }

%---------------------------------------------------------
  \cventry
    {\href{https://ryangroch.github.io/sudoku-v2/}{ryangroch.github.io/sudoku-v2/}} % Job title
    {Sudoku Generator} % Organization
    {March 2022} % Location
    {} % Date(s)
    {
      \begin{cvitems} % Description(s) of tasks/responsibilities
        \item {Used recursive backtracking to write algorithms to generate and solve random Sudoku puzzles.}
        \item {Created a user interface to allow the user to play through the puzzles.}
        \item {Implemented common utilities into the game such as undoing previous moves and taking notes.}
      \end{cvitems}
    }

%---------------------------------------------------------

\end{cventries}
