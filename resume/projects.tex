%-------------------------------------------------------------------------------
%	SECTION TITLE
%-------------------------------------------------------------------------------
\cvsection{Projects}


%-------------------------------------------------------------------------------
%	CONTENT
%-------------------------------------------------------------------------------
\begin{cventries}

%---------------------------------------------------------

  \cventry
  {\href{https://github.com/ryangroch/retrieve-jobs}{github.com/ryangroch/retrieve-jobs}} % Job title
  {Retrieve Jobs} % Organization
  {May 2024 - Present} % Location
  {} % Date(s)
  {
    \begin{cvitems} % Description(s) of tasks/responsibilities
      \item {\textbf{Technologies:} Next.js, TypeScript, React.js, TailwindCSS, Rust}
      \item {Developed a web application enabling users to interact with mainframe held queues by listing, downloading, and purging job outputs via FTP.}
      \item {Implemented seamless deployment to Vercel, Netlify, and Render with zero server configuration, using a custom build-time solution that integrates FTP functionality via a lightweight command-line tool for file transfers.}
      \item {Integrated AES-256 encryption for storage of mainframe credentials in secure, HTTP-only cookies, enabling a safe "stay logged in" feature without storing plaintext passwords.}
      \item {Successfully serving the needs of several dozen users, maintaining consistent uptime with no currently reported issues in production.}
      \item {Developed a desktop version using Rust, reusing the same frontend code as the web app, with FTP calls adapted to the desktop environment.}
    \end{cvitems}
  }

%---------------------------------------------------------
  \cventry
    {\href{https://lightspeedtriggers.com}{lightspeedtriggers.com}} % Job title
    {Ecommerce Store} % Organization
    {January 2023 - Present (in progress)} % Location
    {} % Date(s)
    {
      \begin{cvitems} % Description(s) of tasks/responsibilities
        \item {\textbf{Technologies:} PHP, Laravel, TypeScript, React.js, Stripe API, Linux}
        \item {Developed a full-stack web application, featuring a flexible product configuration that allows users to select from various interchangeable options based on database contents.}
        \item {Integrated Stripe API for one-time payment transactions.}
        \item {Deployed the project to a Linux server via manual SSH.}
        \item {Worked independently to translate requirements into functional features, addressing ongoing changes and feedback to refine the product.}
      \end{cvitems}
    }

%---------------------------------------------------------

% \cventry
% {\href{https://ryan-groch.github.io/image-cropper/}{ryan-groch.github.io/image-cropper/}} % Job title
% {Image Cropper} % Organization
% {March 2023} % Location
% {} % Date(s)
% {
%   \begin{cvitems} % Description(s) of tasks/responsibilities
%     \item {Wrote an algorithm in TypeScript to crop an image in a circular pattern.}
%     \item {Utilized HTML canvas for image processing.}
%     \item {Used GitHub Actions for automatic deployment.}
%   \end{cvitems}
% }

%---------------------------------------------------------
  % \cventry
  %   {\href{https://ryangroch.github.io/othello-ai-v2/}{ryangroch.github.io/othello-ai-v2/}} % Job title
  %   {Othello AI} % Organization
  %   {April 2022 - November 2022} % Location
  %   {} % Date(s)
  %   {
  %     \begin{cvitems} % Description(s) of tasks/responsibilities
  %       \item {Implemented the rules of the Othello game in Typescript, and created a UI with React.js.}
  %       \item {Used the minimax algorithm with alpha-beta pruning to create an AI that demonstrates a reasonable quality of play.}
  %       \item {Made use of web workers and asynchronous programming to avoid freezing the UI as the algorithm performs computations.}
  %       \item {Used GitHub actions to automatically redeploy the app on each new commit.}
  %     \end{cvitems}
  %   }

%---------------------------------------------------------
  % \cventry
  %   {\href{https://ryangroch.github.io/sudoku-v2/}{ryangroch.github.io/sudoku-v2/}} % Job title
  %   {Sudoku Generator} % Organization
  %   {March 2022} % Location
  %   {} % Date(s)
  %   {
  %     \begin{cvitems} % Description(s) of tasks/responsibilities
  %       \item {Used recursive backtracking to write algorithms to generate and solve random Sudoku puzzles.}
  %       \item {Created a user interface to allow the user to play through the puzzles.}
  %       \item {Implemented common utilities into the game such as undoing previous moves and taking notes.}
  %     \end{cvitems}
  %   }

%---------------------------------------------------------

\end{cventries}
